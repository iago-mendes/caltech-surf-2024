\documentclass{../letter}

\addbibresource{refs.bib}

\renewcommand{\v}[1]{\boldsymbol{#1}}
\newcommand{\vv}[1]{\underline{\boldsymbol{#1}}}

\begin{document}
	\title
		[Caltech SURF Proposal]
		{Iago Mendes\fnote{ibrazmen@oberlin.edu}}
		{Control of Black Hole Parameters for Binary Evolutions}

	\section{Introduction \& Background}

	In the early twentieth century, Einstein revolutionized the study of gravity by connecting spacetime geometry with physical dynamics. As John Wheeler says, ``Spacetime tells matter how to move; matter tells spacetime how to curve'' \cite{Wheeler}. Being a highly complex theory, many problems of interest only have analytic solutions in special cases with symmetry. In this context, Numerical Relativity emerged as an essential field to solve these problems numerically, allowing us to explore general cases that can be found in the universe. Specifically, simulations of Binary Black Holes (BBH) became very important as gravitational wave detectors were developed, needing to use numerical results to identify and characterize signals in their data \cite{LIGO}.

	Famously, the Einstein equations relate the curvature of spacetime to the stress-energy of matter, forming a system of ten nonlinear partial differential equations (PDEs). With the 3+1 formalism, we can rearrange these equations so that spacetime is described by spacelike three-dimensional slices of constant time \cite{Alcubierre}. In doing so, we find that four out of the ten equations do not involve time derivatives, implying that they are constraints that must be satisfied at all times. The remaining six equations describe an evolution of the constraint-satisfying fields. Using this formalism, the Spectral Einstein Code (SpEC) \cite{SpEC} runs BBH simulations by first finding initial data and then running an evolution on them. Over time, as SpEC faced more challenging BBH with high mass ratios and spins, several improvements had to be made to the initial data techniques, which are summarized in \cite{Serguei}.
	
	To enforce quasi-equilibrium conditions, SpEC uses the extended conformal thin-sandwich (XCTS) decomposition \cite{Serguei}, which transforms the constraint PDEs into a system of elliptic PDEs. Similar to any elliptic PDE problem, this system requires boundary conditions. Hence, SpEC has to choose \textit{free} data and impose them at the boundary before the elliptic solver can be used on the XCTS equations. From the resulting fields, one can measure \textit{physical} parameters of the simulation, such as the black holes masses and spins, the total linear momentum and the center of mass. Even though we wish to control these physical quantities, it is not possible to measure them before the elliptic solve. Therefore, SpEC has to iterate over different choices of free data, trying to find the ones that result in the desired physical parameters.
	
	Despite its success in BBH simulations, SpEC shows its limitations in more challenging problems, such as binary neutron star mergers and BBH with extreme configurations. In this context, SpECTRE \cite{SpECTRE} was created as a codebase that follows a parallelism model and aims to be more scalable \cite{Kidder}. Previous work has already shown that SpECTRE can be faster and more accurate than SpEC when performing similar tasks due to its use of parallelism \cite{Vu}. This will be especially needed for the upcoming gravitational wave detectors with higher sensitivity, such as the Cosmic Explorer, the Einstein Telescope and LISA.
	
	As part of an effort to allow researchers to simulate BBH in SpECTRE, an initial data procedure similar to the one in SpEC needs to be completed. This is greatly benefitted by a scalable elliptic solver that was recently developed \cite{Vu}, which can now be used to solve the XCTS equations. Be as it may, SpECTRE does not currently have an iterative method to adjust the free data used as boundary conditions. As described in more detail in the next sections, this is the problem that we aim to address with this research project.

	\section{Objectives}

	The main objective of this project is to implement the control of \textit{physical} parameters of BBH simulations in SpECTRE. Specifically, during the initial data procedure, we need to control
	\begin{enumerate}
		\item the black hole masses $M_A$ and $M_B$;
		\item the black hole spins $\v \chi_A$ and $\v \chi_B$;
		\item the center of mass $\v C$; and
		\item the total linear momentum $\v P$.
	\end{enumerate}
	
	When running a simulation, users specify the masses $M_{A,B}^*$ and spins $\v \chi_{A,B}^*$ of the black holes they wish to simulate. Therefore, we want to drive $M_{A,B} - M_{A,B}^*$ and $\v \chi_{A,B} - \v \chi_{A,B}^*$ to zero. Similarly, we want to drive $\v C$ and $\v P$ to zero in order to minimize any drifts of the orbit for long simulations. As mentioned before, we can only measure these physical parameters after the elliptic solve of the XCTS equations. Similar to SpEC, the goal here is to apply an iterative scheme on the choice of boundary conditions, which is explained in further detail in the next section.

	\section{Approach}

	Before running the elliptic solver on the XCTS equations, we have to choose \textit{free} data that will be used as boundary conditions. Specifically, we have to set
	\begin{enumerate}
		\item the apparent horizon radii $r_A$ and $r_B$;
		\item the apparent horizon angular frequencies $\v \Omega_A$ and $\v \Omega_B$;
		\item the first apparent horizon center $\v c_A$; and
		\item a constant velocity $\v v_0$ used on the shift boundary condition.
	\end{enumerate}

	We can iterate through the choices of these free data simultaneously. While the overall idea of this iteration is described below, many technical details were omitted in the interest of conciseness. More details and explanations can be found in sections 2.3 and 2.5 of \cite{Serguei}.

	In order to control the physical parameters $M_{A,B}$ and $\v\chi_{A,B}$, we need to determine the free data $r_{A,B}$ and $\v\Omega_{A,B}$. Let the choice of these eight values be represented as
	\begin{equation}
		\vv u = (r_A, r_B, \v\Omega_A, \v\Omega_B)
	\end{equation}
	\cite{Serguei}. Also, let the difference between the current and desired physical parameters be represented as
	\begin{equation}
		\vv F(\vv u) = (M_A - M_A^*, M_B - M_B^*, \v \chi_A - \v \chi_A^*, \v \chi_B - \v \chi_B^*)
	\end{equation}
	 \cite{Serguei}. Since we want to find the values of $\vv u$ such that $\vv F(\vv u) = 0$, we can perform root-finding iterations using a quasi-Newton method:
	\begin{equation}
		\vv u_{k+1} = \vv u_k - J_k^{-1} \vv F(\vv u_k)
	\end{equation}
	\cite{Serguei}, where the Jacobian $J_k$ is constructed once and then updated using Broyden's method:
	\begin{equation}
		J_k = J_{k-1} + \vv F(\v u_k) \frac{\Delta \vv u_k^T}{||\Delta \vv u_k||}
	\end{equation}
	\cite{Serguei}, where we use $\Delta x_k = x_k - x_{k-1}$ for any variable $x$ from now on. We must continue this iteration until we reach the target residuals.

	Additionally, in order to control the center of mass $\v C$, we fix the distance between the apparent horizon centers with $\v c_A - \v c_B = \v D$, where $\v D$ is specified by users \cite{Serguei}. With this, we can iteratively find the value of $\v c_A$ that makes $\v C = 0$ using
	\begin{equation}
		\v c_{A,k+1} = \v c_{A,k} - \v C_k - \frac{M_{A,k}  \Delta M_{B,k} - M_{B,k} \Delta M_{A,k}}{(M_{A,k} + M_{B,k})^2} \v D
	\end{equation}
	\cite{Serguei}.

	Finally, in order to control the total linear momentum $\v P$, we can add a constant velocity $\v v_0$ to the outer boundary condition of the shift, which will affect the overall motion of the system, resulting in a different momentum \cite{Serguei}. Similar to before, we can now iterate to find values of $\v v_0$ that make $\v P = 0$ using
	\begin{equation}
			\v v_{0,k+1} = \v v_{0,k} - \frac{\v P_k}{M_k} + \Delta M_k (\v v_{0,k} + \v\Omega_0 \times \v c_{A,k}) - \v\Omega_0 \times \delta \v c_{A,k} - \frac{\Delta M_{B,k}}{M_k} \v\Omega_0 \times \v D.
		\end{equation}
	\cite{Serguei}, where $M_k = M_{A,k} + M_{B,k}$, $\v \Omega_0$ is the user-specified initial orbit angle, and $\delta \v c_{A,k}$ is a small perturbation of $\v c_{A,k}$.

	\section{Work Plan}

	I outline below a tentative schedule for each week of the SURF program. In the action plan column, I describe the research project milestones and the program deadlines. Apart from that, I also plan to attend the weekly seminars, workshops, social activities and other events.

	\begin{center}
		\setlength{\tabcolsep}{0.5cm}
		\def\arraystretch{1.5}
		\begin{tabular}{cp{0.7\textwidth}}
			\hline
			\textbf{Week} & \textbf{Action Plan} \\ \hline
			1 & Get used to SpECTRE development by following its \href{https://spectre-code.org/dev_guide.html}{developer guidelines}. Review relevant parts of \cite{Serguei} and related topics. \\
			2 & Understand the initial data code infrastructure and how to interface with the elliptic solver. Prepare code for the iterations. \\
			3 & Implement iterations for $r_{A,B}$ and $\v \Omega_{A,B}$. \\
			4 & Test previous implementation. Write the first interim report. \\
			5 & Implement iterations for $\v c_A$. \\
			6 & Test previous implementation. \\
			7 & Implement iterations for $\v v_0$. \\
			8 & Test previous implementation. Write the second interim report. \\
			9 & Write technical report and abstract. Prepare for oral presentation. \\
			10 & Present at a SURF Seminar Day. Conclude any pending work. \\ \hline
		\end{tabular}
	\end{center}

	\newpage

	\section{References}

	This proposal was written under the guidance of my prospective Caltech mentor, Nils L. Vu\fnote{nilsvu@caltech.edu}, who indicated me to \cite{Serguei} as a starting point and gave me valuable suggestions to improve the motivation for Numerical Relativity and SpECTRE in the introduction section. Additionally, I had substantial conversations with my current Oberlin advisor, Robert Owen\fnote{rowen@oberlin.edu}, who helped me understand the theory behind the initial data procedure in SpEC.

	\printbibliography[heading=none]
\end{document}
